\documentclass[12pt]{article}
\usepackage[pdftex,pagebackref,letterpaper=true,colorlinks=true,pdfpagemode=none,urlcolor=blue,linkcolor=blue,citecolor=blue,pdfstartview=FitH]{hyperref}

\usepackage{amsmath,amsfonts}
\usepackage{graphicx}
\usepackage{color}


\setlength{\oddsidemargin}{0pt}
\setlength{\evensidemargin}{0pt}
\setlength{\textwidth}{6.0in}
\setlength{\topmargin}{0in}
\setlength{\textheight}{8.5in}

\setlength{\parindent}{0in}
\setlength{\parskip}{5px}

\input{macrosblog.tex}


%%%%%%%%%%%%%%%%%%%%%%%%%%%%%%%%%%%%%%%%%%%
% pacotes utilizados
\usepackage[T1]{fontenc}
\usepackage[utf8]{inputenc}
\usepackage{lmodern}
\usepackage{hyperref}
\usepackage{graphicx}
\usepackage[portuguese]{babel}
\usepackage{amsfonts}
\usepackage[usenames,dvipsnames]{xcolor}
\usepackage{mathtools}
\usepackage{amssymb} % alguns simbolos matematicos
\usepackage{mathrsfs} % letras cursivas
\usepackage{listings} % coding examples in latex file
\usepackage{xcolor} % changing colors
\usepackage{amsthm} % definir estilo do teorema
\usepackage[hang,flushmargin]{footmisc} % remove footnote's identation
\usepackage{cancel} % para poder colocar o tracinho de cancelamento
\usepackage{tikz} % to draw Venn's diagrams and Real line
\usepackage{subfiles} % to make multi .tex files into one single pdf
\usepackage{amsmath}

\usetikzlibrary{arrows} % to write real lines:

% dark theme no pdf
%\pagecolor[rgb]{0.1,0.1,0.1} %black
%\color[rgb]{0.9,0.9,0.9} %grey

% criando o modelo de definicoes/teoremas/fatos/demonstracoes
\theoremstyle{definition}

\newtheoremstyle{break}% name
  {10pt}%         Space above, empty = `usual value'
  {10pt}%         Space below
  {}% Body font
  {}%         Indent amount (empty = no indent, \parindent = para indent)
  {\bfseries}% Thm head font
  {}%        Punctuation after thm head
  {\newline}% Space after thm head: \newline = linebreak
  {}%         Thm head spec

\theoremstyle{break}

% definindo as categorias de formalidade
\newtheorem{definition}{Definição}[section]
\newtheorem{fact}{Fato}[section]
\newtheorem{demonstration}{Demonstração}[section]
\newtheorem{theorem}{Teorema}
\newtheorem{proposition}{Proposição}

% tirando identação dos paragrafos
\setlength{\parindent}{0ex}

% dedicatoria 
% Source: http://www.tug.org/pipermail/texhax/2010-June/
\newenvironment{dedication}
{
   \cleardoublepage
   \thispagestyle{empty}
   \vspace*{\stretch{1}}
   \hfill\begin{minipage}[t]{0.66\textwidth}
   \raggedright
}
{
   \end{minipage}
   \vspace*{\stretch{3}}
   \clearpage
}

\makeatletter
\renewcommand{\@chapapp}{}% Not necessary...
\newenvironment{chapquote}[2][2em]
  {\setlength{\@tempdima}{#1}%
   \def\chapquote@author{#2}%
   \parshape 1 \@tempdima \dimexpr\textwidth-2\@tempdima\relax%
   \itshape}
  {\par\normalfont\hfill--\ \chapquote@author\hspace*{\@tempdima}\par\bigskip}
\makeatother

%%%%%%%%%%%%%%%%%%%%%%%%%%%%%%%%%%%%%%%%%%%

\begin{document}


Até agora, vimos quantas listas de tamanho $k$ podemos fazer com todos os $n$ elementos de um conjunto qualquer. Agora, vamos expandir essa linha de pensamento para um tópico correlato: "Quantos subconjuntos podem ser feitos se selecionarmos $k$ elementos de um conjunto de $n$ elementos?". 
\\~\\
Algum leitor de pensamento rápido pode pensar que se trata exatamente do mesmo problema. Contudo, não é o caso. A diferença reside nas propriedades de uma \textbf{lista} e de um \textbf{conjunto}.
\\~\\
Uma lista leva em consideração a ordem dos seus elementos, ou seja, $(a,b) \neq (b,a)$ enquanto os conjuntos só se preocupam com os valores dos seus elementos $\{a,b\} = \{b,a\}$. Para prosseguirmos vamos definir essa tarefa de "contar subconjuntos" \ usando uma notação.
\\~\\
\textbf{Comentário}: Estranhamente, o professor não dá um nome para o conceito que ele vai definir agora. Nós achamos por bem usar o nome que os materiais didáticos do Brasil utilizam.

\begin{definition}[Combinação]
Se $k,n \in \mathbb{Z}$, então $n \choose k$, ou alternativamente, $C(n,k)$, denota o número de subconjuntos que podem ser feitos escolhendo $k$ elementos de um conjunto com $n$ elementos. Lemos $n \choose k$ como "de $n$ escolhemos $k$".
\end{definition}

\textbf{Comentário}: Aqui já temos algo para guardar na mente. Se tivermos um conjunto com $n$ elementos, sempre será verdade que $P(n,k) \geqslant $ $n \choose k$ ou, na outra notação, $P(n,k) \geqslant C(n,k)$. Ou seja, sempre será verdade que os arranjos serão maiores ou iguais que as combinações para quaisquer $n$ e $k$.
\\~\\
Já definimos o que queremos dizer quando escrevemos $n \choose k$ e sabemos bem qual problema queremos responder com essa notação nova. Precisamos de uma fórmula que nos permita encontrar a solução de quaisquer valores de $n$ e $k$. Para tanto, vamos começar com um exemplo prático: $5 \choose 3$.
\\~\\
Nós já sabemos que se fossem listas ao invés de subconjuntos, teríamos $P(5,3) = \frac{5!}{(5 - 3)!} = 60$ listas possíveis. Também já sabemos que $P(5,3) \geqslant $ $5 \choose 3$.
\\~\\
O pensamento para chegarmos no nosso resultado é o seguinte: "Quantas listas podemos formar com cada subconjunto de 3 elementos?". A resposta é obtida pela aplicação do princípio da multiplicação. Isso nos dá $3! = 6$ listas possíveis para cada subconjunto de 3 elementos.
\\~\\
Agora que sabemos que teremos 6 listas para cada subconjunto formado por 3 elementos e também sabemos que podemos formar um total de 60 listas diferentes. Basta dividirmos as 60 listas por 6 que chegaremos no total de 10 subconjuntos possíveis.
\\~\\
\textbf{Dica}: Na página 89 o professor monta uma tabela com todos os subconjuntos e as listas que falamos aqui.
\\~\\
De maneira mais abstrata, o que fizemos foi dividir a k-permutação $P(5,3)$ por $3!$. Ou seja, podemos saber quantos subconjuntos de tamanho $k$ podem ser formados com $n$ elementos pela equação abaixo.

\begin{fact}[Cálculo das Combinações]
Se $0 \leq k \leq n$, então:
\begin{center}
 \Large ${n \choose k} = \frac{n!}{k!(n-k)!}$
\end{center}
\end{fact}

Se $k > n$ então $n \choose k$ $= 0$. Seria como perguntar algo como "quantos subconjuntos de 5 elementos podemos fazer com 3 elementos? Isso mesmo, zero conjuntos.


\end{document}
